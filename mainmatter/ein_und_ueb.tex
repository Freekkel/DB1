\chapter{Einleitung und Übersicht}
\textit{Datenbankverwaltungssysteme} (DBMS) gewinnen dank der digitalen Vernetzung und Kommunikation immer mehr an Bedeutung und sind heutzutage nicht mehr aus unserem Alltag wegzudenken. Zwar bemerken viele den Einsatz eines DBMS nicht jedoch sind diese Systeme überall

\section{Empfohlene Lektüre:}
Für die Vorlesung wird das folgende Buch dringend empfohlen, da sich die Vorlesung an diesem orientieren wird. \\ vertreten z.B. in Versicherungen, Banken, Universitäten (welche das Hauptbeispiel für diese Vorlesung bildet) noch bei vermeintlich nicht aufwendigen Blogs wie einem einfachem Blog.\\
Ein Datenbankverwaltungssystem besteht aus einer Menge von \textit{Daten} und dem \textit{Datenbankverwaltungssystem} 
\begin{itemize}
	\item Die \textit{Daten} werden auch als Datenbasis bezeichnet welche in einer gewissen Beziehung zueinander stehen. 
	\item Die Programme zum Zugriff, Modifikation und Kontrolle genutzt werden, werden als \textit{Datenbankverwaltungssystem} bezeichnet
\end{itemize}
Diese Trennung wird häufig unterlassen, um eine Verwirrung zu vermeiden. Stattdessen werden Datenbankverwaltungssysteme (auch Datenbanksysteme) als Kombination der vorher genannten Unterschiede bezeichnet. 

\section{Motivation für ein DBMS}
Es gibt einige Probleme wenn Firmen oder Organisationen auf ein DBMS verzichten müssten und auf Papier (Karteikarten, Akten, \dots) oder separate Dateien zurück greifen müssten. 
\begin{description}
  \item[Redundanz und Inkonsistenz:] \hfill\\
   Wenn Daten isoliert gehalten werden, müssen diese mehrfach vorhanden sein (Zweigstelle einer Firma) und wenn Informationen in einer dieser Dateien geändert werden, geschieht dies lediglich in dieser einen Dateien, Hierdurch entstehen, aufgrund der vorhandenen Redundanz, Inkonsistenzen in den Informationen. 
  \item[Beschränkte Zugriffsmöglichkeiten:] \hfill \\
  	Es ist nahezu unmöglich separierte Dateien miteinander zu verknüpfen und logische Abhängigkeiten zu generieren. Mit einem DBMS können Informationen einer Firma/Organisation einheitlich modelliert werden (\text{Datenmodell}) . Dadurch lassen sich die Daten auf viele Arten verknüpfen. 
  \item[Probleme beim Mehrbenutzerbetrieb:] \hfill \\
  	Dateisysteme bieten nur wenige Kontrollmechanismen für den Mehrbenutzerbetrieb. Das bedeutet, falls mehrere Benutzer an einer Datei arbeiten, wird bei jedem Speicher Vorgang alle vorherigen Änderungen überschrieben. Dies nennt man ein \glqq{}lost update\grqq{}. DBMS bieten eine Mehrbenutzerkontrolle, welche solche Anomalien erkennt und verhindert. 
  \item[Verlust von Daten:] \hfill \\ 
  	Bei isolierten Daten wird die Wiederherstellung eines konsistenten Zustandes äußerst schwierig. Datenbankverwaltungssysteme haben Möglichkeiten diese Daten wieder herzustellen.
  \item[Integritätsverletzung:]\hfill \\ 
  	Diese \textit{Abhängigkeitsverletzungen} können auftreten, wenn Daten und Informationen isoliert bearbeitet und betrachtet werden. Die Überprüfung auf Integritätsverletzungen sollte vom System geprüft und bei Bedarf abgelehnt werden. DBMS führen Transaktionen nur aus, wenn sie die Datenbasis in eine konsistenten Zustand überführen. 
  \item[Sicherheitsprobleme:] \hfill \\
  	Nicht jeder sollte Zugriff auf alle gespeicherten Daten haben, sondern lediglich auf bestimmte Bereiche und Daten, die unbedingt notwendig sind. Auch sollten nur bestimmte Leute das \textit{Privileg} haben Daten zu verändern oder gar zu löschen. DBMS Systeme haben ausgefeilte Rechteverwaltungen für Benutzer und Benutzergruppen.. 
  \item[hohe Entwicklungskosten für Anwendungsprogramme:] \hfill \\
  	Anwendungsentwickler müssen sich meist mit einem Teil der vorher genannten Probleme auseinandersetzen um Informationen zu speichern und zu verwalten. Ein DBMS bietet eine einfache und erprobte Möglichkeit diese Probleme direkt zu erledigen.
\end{description}

\section{Die Abstraktionsebenen eines Datenbanksystems}
\begin{figure}[h!]
\centering
\includegraphics[width=0.45\linewidth]{./mainmatter/pics/sichten}
\caption[Abstraktionsebenen]{Drei Abstraktionsebenen eines Datenbanksystems}
\label{fig:sichten}
\end{figure}
Bei einem DBMS unterscheidet man drei Abstraktionsebenen(siehe Abbildung \ref{fig:sichten}).
\begin{description}
\item[Die physische Ebene:] Auf dieser Ebene wird festgelegt, wie die Daten gespeichert sind. Im Allgemeinen sind die Daten auf einem Speichermedium (meistens als Plattenspeicher realisiert) abgelegt. 
\item[Die logische Ebene:] Auf der logischen Ebene wird in einem sogenannten \textit{Datenbankschema} festgelegt, welche Daten abgespeichert sind. 
\item[Die Sichten:] Während das Datenbankschema der logischen Ebene ein integriertes Modell der gesamten Informationsmenge des jeweiligen Anwendungsbereichs (z.B. des gesamten Unternehmens) darstellt, werden in den Sichten Teilmenge der Informationen bereitgestellt. Diese Sichten sind auf Benutzer bzw. Benutzergruppen zugeschnitten, die lediglich diese Sichten benötigen. Denn nicht jeder Benutzer benötigt alle Daten die existieren.
\end{description}

\section{Datenunabhängigkeit}
Durch die drei Ebenen wird eine Datenunabhängigkeit erreicht, welche zu wohl definierten Schnittstellen führt, ermöglicht es die Realisierung der Datenbank zu variieren ohne die Benutzer der Schnittstellen in Mitleidenschaft zu ziehen. Aus den drei Schichten ergeben sich zwei Stufen der Datenunabhängigkeit im DBMS

\begin{description}

\item[Physische Datenunabhängigkeit:] Die Modifikation der physischen Speicherstruktur belässt die logische Ebene (das Datenbankschema) invariant. Z.B. erlauben fast alle Datenbanksysteme das nachträgliche Anlegen eines Indexes, um die Datenbankobjekte schneller finden zu können. Dies darf keinen Einfluss au bereits existierende Anwendungen in der logischen Ebene haben. Es sollte lediglich die Effizienz positiv beeinflussen.
\item[Logische Datenunabhängigkeit:] Die Anwendungen nehmen stets Bezug auf die logische Struktur der Datenbasis. Bei Änderungen der logischen Ebene (des Datenbankschemas) können diese in den Sichten vor dem Anwender verborgen werden. Dadurch wird zu einem gewissen Grad eine logische Datenbankunabhängigkeit erzielt. 

\end{description}

Die heutigen Datenbanksysteme erfüllen mindestens die physische Datenbankunabhängigkeit. Die logische Datenunabhängigkeit kann schon rein konzeptuell nur für einfachste Modifikationen des Datenbankschemas gewährleistet werden. 

\section{Datenmodellierung}

\begin{figure}[h!]
\centering
\includegraphics[width=0.55\linewidth]{./mainmatter/pics/modell}
\caption{Datenmodellierung}
\label{fig:modell}
\end{figure}

Datenbankverwaltungssysteme basiert auf einem so genannten Datenmodell. Das Datenmodell stellt die Infrastruktur für die Modellierung der realen Welt zur Verfügung. Das Datenmodell beinhaltet die Möglichkeit zur 

\begin{itemize}
\item Beschreibung der Datenobjekte und zur
\item Festlegung der anwendbaren Operatoren und deren Wirkung
\end{itemize}

Das Datenmodell legt somit die generischen Strukturen und Operatoren fest, die man zur Modellierung einer bestimmten Anwendung ausnutzen kann. Eine Programmiersprache legt die Typkonstruktoren und Sprachkonstrukte fest, mit deren Hilfe man spezifische Anwendungsprogramme realisiert. 

\begin{tabbing}
\qquad \= Das Datenmodell besteht demnach aus zwei Teilsprachen: 
\end{tabbing}

\begin{enumerate}
\item der \textit{Datendefinitionssprache} (engl. \textit{Data Definition Langauge}, DDL)
\item der \textit{Datenmanipulationssprache} (engl. \textit{Data Manipulation Langauge}, DML)
\end{enumerate}

Die DDL wird benutzt, um die Struktur der abzuspeichernden Datenobjekte zu beschreiben. Die Strukturbeschreibung aller Datenobjekte des betrachteten Anwendungsbereichs nennt man das Datenbankschema. 

\begin{tabbing}
\qquad \= Die Datenmanipulationssprache (DML) besteht aus
\end{tabbing}

\begin{itemize}
\item der \textit{Anfragesprache} (engl. \textit{Query Language})
\item der \glqq eigentlichen\grqq Datenmanipulationssprache zur Änderung von abgespeicherten Datenobjekten, zum Einfügen von Daten und zum Löschen von gespeicherten Daten.
\end{itemize}

Die DML (einschließlich der Anfragesprache) kann in zwei unterschiedlichen Arten genutzt werden:

\begin{itemize}
\item \textit{interaktiv}, indem DML-Kommandos direkt am Arbeitsplatzrechner (oder Terminal) eingegeben werden
\item in einem Programm einer höheren Programmiersprache, das \glqq eingebettet \grqq DML-Kommandos enthält.
\end{itemize}

\section{Logische Datenmodelle}
\begin{itemize}
\item Netzwerkmodell
\item Hierachisches Datenmodell
\item Relationales Datenmodell
\item XMl Schema
\item Objektorintiertes Datenmodell
\begin{itemize}
\item Objektrealtionales Schema
\end{itemize}
\item Deduktives Datenmodell
\end{itemize}
