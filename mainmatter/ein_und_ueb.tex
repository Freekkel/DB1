\chapter{Einleitung und Übersicht}
\textit{Datenbankverwaltungssysteme} (DBMS) gewinnen dank der digitalen Vernetzung und Kommunikation immer mehr an Bedeutung und sind heutzutage nicht mehr aus unserem Alltag wegzudenken. Zwar bemerken viele den Einsatz eines DBMS nicht jedoch sind diese Systeme überall vertreten z.B. in Versicherungen, Banken, Universitäten (welche das Hauptbeispiel für diese Vorlesung bildet) noch bei vermeintlich nicht aufwendigen Blogs wie einem einfachem Blog.\\
Ein Datenbankverwaltungssystem besteht aus einer Menge von \textit{Daten} und dem \textit{Datenbankverwaltungssystem} 
\begin{itemize}
	\item Die \textit{Daten} werden auch als Datenbasis bezeichnet welche in einer gewissen Beziehung zueinander stehen. 
	\item Die Programme zum Zugriff, Modifikation und Kontrolle genutzt werden, werden als \textit{Datenbankverwaltungssystem} bezeichnet
\end{itemize}
Diese Trennung wird häufig unterlassen, um eine Verwirrung zu vermeiden. Stattdessen werden Datenbankverwaltungssysteme ( auch Datenbanksysteme) als Kombination der vorher genannten Unterschiede bezeichnet. 
\section{Motivation für ein DBMS}
Es gibt einige Probleme wenn Firmen oder Organisationen auf ein DBMS verzichten müssten und auf Papier (Karteikarten, Akten, \dots) oder separate Dateien zurück greifen müssten. 
\begin{description}
  \item[Redundanz und Inkonsistenz:] \hfill\\
   Wenn Daten isoliert gehalten werden, müssen diese mehrfach vorhanden sein (Zweigstelle einer Firma) und wenn Informationen in einer dieser Dateien geändert werden, geschieht dies lediglich in dieser einen Dateien, Hierdurch entstehen, aufgrund der vorhandenen Redundanz, Inkonsistenzen in den Informationen. 
  \item[Beschränkte Zugriffsmöglichkeiten:] \hfill \\
  	Es ist nahezu unmöglich separierte Dateien miteinander zu verknüpfen und logische Abhängigkeiten zu generieren. Mit einem DBMS können Informationen einer Firma/Organisation einheitlich modelliert werden (\text{Datenmodell}) . Dadurch lassen sich die Daten auf viele Arten verknüpfen. 
  \item[Probleme beim Mehrbenutzerbetrieb:] \hfill \\
  	Dateisysteme bieten nur wenige Kontrollmechanismen für den Mehrbenutzerbetrieb. Das bedeutet, falls mehrere Benutzer an einer Datei arbeiten, wird bei jedem Speicher Vorgang alle vorherigen Änderungen überschrieben. Dies nennt man ein \glqq{}lost update\grqq{}. DBMS bieten eine Mehrbenutzerkontrolle, welche solche Anomalien erkennt und verhindert. 
  \item[Verlust von Daten:] \hfill \\ 
  	Bei isolierten Daten wird die Wiederherstellung eines konsistenten Zustandes äußerst schwierig. Datenbankverwaltungssysteme haben Möglichkeiten diese Daten wieder herzustellen.
  \item[Integritätsverletzung:]\hfill \\ 
  	Diese \textit{Abhängigkeitsverletzungen} können auftreten, wenn Daten und Informationen isoliert bearbeitet und betrachtet werden. Die Überprüfung auf Integritätsverletzungen sollte vom System geprüft und bei Bedarf abgelehnt werden. DBMS führen Transaktionen nur aus, wenn sie die Datenbasis in eine konsistenten Zustand überführen. 
  \item[Sicherheitsprobleme:] \hfill \\
  	Nicht jeder sollte Zugriff auf alle gespeicherten Daten haben, sondern lediglich auf bestimmte Bereiche und Daten, die unbedingt notwendig sind. Auch sollten nur bestimmte Leute das \textit{Privileg} haben Daten zu verändern oder gar zu löschen. DBMS Systeme haben ausgefeilte Rechteverwaltungen für Benutzer und Benutzergruppen.. 
  \item[hohe Entwicklungskosten für Anwendungsprogramme:] \hfill \\
  	Anwendungsentwickler müssen sich meist mit einem Teil der vorher genannten Probleme auseinandersetzen um Informationen zu speichern und zu verwalten. Ein DBMS bietet eine einfache und erprobte Möglichkeit diese Probleme direkt zu erledigen.
\end{description}