\chapter{Datenintegrität}
\section{Allgemein}
Datenintegrität meint semantische Integritätsbedingungen, die sich aus der zu modellierenden Miniwelt ableiten $\rightarrow$ Modell soll Konsistenz der Daten gewährleisten\\
Es gibt statische (Wertebereiche mit check zu realisieren) und dynamische (Update-Prüfung mit Trigger zu realisieren) Integritätsbedingungen\\
Schlüssel haben Unique-Charakter, Kardinalitäten legen Beziehungen fest z. B. anhand von Fremdschlüsseln, jedes Attribut hat Domäne z. B. 5 Ziffern für Matrikelnummer\\
Referentielle Integrität = ein Fremdschlüssel von R enthält entweder Null-Werte oder einen in S vorkommenden Primärschlüssel (auch mehrere Attribute als Schlüssel denkbar)\\
\begin{itemize}
\item in SQL durch unique, primary key $\rightarrow$ not null, foreign key $\rightarrow$ unique foreign key bezeichnet 1:1-Beziehung
\item cascade $\rightarrow$ Änderungen Primärschlüssel bedingen Änderungen Fremdschlüssel, on delete cascade $\rightarrow$ zahlreiche Löschoperationen, on update/delete set null $\rightarrow$ keine kaskadierenden Löschoperationen
\end{itemize}
\section{Trigger}
Diese ermöglichen das eine Datenbank bei Änderungen de Inhalts oder Schemas selbst aktiv wird und macht somit eine Datendank zu einer aktiven Datenbank. 
\begin{lstlisting}
create trigger keineDegradierung before update on Professoren
for each row
when (old.Rang is not null)
begin
	if :old.Rang = 'C3' and :new.Rang = 'C2' then
		:new.Rang := 'C3';
	end if;
	if :old.Rang = 'C4' then
		:new.Rang := 'C4';
	end if;
	if :new.Rang is null then
		:new.Rang := :old.Rang;
	end if;
end
\end{lstlisting}