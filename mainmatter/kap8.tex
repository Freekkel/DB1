\chapter{Anfragebearbeitung}
\section{Aquivalenzen in der relationalen Algebra}
\begin{enumerate}
\item Join, Vereinigung, Schnitt und Kreuzprodukt sind kommutativ und assoziativ
\item Selektionen sind untereinander vertauschbar
\item $\wedge$ in einer Selektionsbedingung können in mehrere Selektionen aufgebrochen bzw. nacheinander ausgeführte Selektionen können durch $\wedge$ zusammengefügt werden. 
\item Geschachtelte Projektionen können auf die äußere beschränkt werden.
\subitem Dabei gilt es zu beachten, dass die Projektion und alle vorherigen eine Submenge des Schemas sein müssen.
\item Eine Selekteion kann an einer Projektion \grqq vorbeigeschoben\glqq ~ werden, falls die Projektion keine Attribute aus der Selektionsbedingung entfernt
\item Eine selektion kann an einer Joinoperation (oder Kreuzprodukt) vorbeigeschoben werden, falls sie nur Attribute eines der beiden Join-Argumente verwendet. 
\item 
\item Selektion auf eine fertige Mengenoperation (Vereinigung, Schnitt und Differenz) = Selektion auf teile und dann Mengenoperation
\item Bei Projektionen nur bei Vereinigung möglich
\item Selektion + Kreuzprodukt = Join mit Selektionsbedingung. 
\item DeMorgans Gesetz: 
\[\neg(p_1 \vee p_2) = \neg p_1 \wedge \neg p_2 \]
\[\neg(p_1 \wedge p_2) = \neg p_1 \vee \neg p_2\]
\end{enumerate}
\section{Anwendung der Transformationsregeln}
\begin{enumerate}
\item Aufbrechen von Selektionen
\item Verschieben der Selektionen soweit wie möglich nach unten im Operatorbaum
\item Zusammenfassen von Selektionen und Kreuzprodukten zu Joins
\item Bestimmung der Reihenfolge der Joins in der Form, dass möglichst kleine Zwischenergebnisse entstehen.
\item unter Umständen Einfügen von Projektionen
\item Verschieben der Projektionen soweit wie möglich nach unten im Operatorbaum
\end{enumerate}