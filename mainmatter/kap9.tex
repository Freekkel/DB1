\chapter{Transaktionsverwaltung}
Grundlegende Anforderungen an Transkationen:
\begin{enumerate}
\item Recovery, d.h. die Behebung von eingetretenen, oft unvermeidbaren Fehlersituationen
\item Synchronisation von mehreren gleichzeitig auf der Datenbank ablaufenden Transaktionen. 
\end{enumerate}
\section{Operationen auf Transaktionsebene}
\begin{description}
\item[begin of transaction (BOT)] Mit diesem Befehl wird der Beginn einer Transaktion dargestellt.
\item[commit] Hierdurch wrid die Beendigung der Transaktion eingeleitet. Alle Änderungen der Datenbasis werden durch diesen Befehl \textbf{festgeschrieben}.
\item[abort] Dieser Befehl führt zu einem Selbstabbruch der Transaktion. Das Datenbanksystem stellt sicher, dass der ursprüngliche Zustand wieder erreicht wird. 
\item[define savepoint] An diesem Punkt in einer Transaktion wird ein Sicherungspunkt angelegt, an dem später wieder eingestiegen werden kann. Die Änderungen werden aber noch nicht in die Datenbank geschrieben. 
\item[backup transaction] Dieser Befehl dient dazu, die noch aktive Transaktion auf den jüngsten savepoint zurück zu setzen und dort weiter zu arbeiten.
\end{description}
\section{Eigenschaften von Transaktionen}
\begin{description}
\item[Atomicity (Atomarität)] Alles oder nichts Prinzip. Die Transaktion ist nicht weiter zerlegbar und wird kompleet oder gar nicht ausgeführt.
\item[Consistency] Eine Transaktion einen Konsistenten Datenbasiszustand, ansonsten wird diese zurückgesetzt. Zwischenzustand darf inkonsistent sein, sonst nie. 
\item[Isolation] Keine Beeinflussung von parallel Laufenden Transaktionen. Alle anderen parallel ausgeführten Transaktionen bzw. deren Effekte dürfen nicht sichtbar sein.
\item[Durability (Dauerhaftigkeit)] Die Wirkung einer erfolgreich abgeschlossenen Transaktion bleibt dauerhaft in der Datenbank erhalten. Nur eine weitere Transaktion kann diese rückgängig machen. 
\end{description}
\section{Transaktionsverwaltung in SQL}
\begin{description}
\item[commit work] Änderungen werden comitted keine Rückmeldung
\item[rollback work] Alle Änderungen werden zurück gesetzt. Rückmeldung vorgeschrieben. 
\end{description}
\section{Zustandsübergänge einer Transaktion}
\begin{description}
\item[potentiell] Codiert und wartet auf die Ausführung
\item[aktiv] wird aktuell ausgeführt. 
\item[wartend] Bei Überlast in wartenden Zustand verdrängt. 
\item[abgeschlossen] Durch commit beendete aktive Transaktion.
\item[persistent] Die Änderungen wurden in die Datenbank geschrieben
\item[gescheitert] Abbruch führt in diesen Zustand. 
\item[wiederholbar] Falls Transaktion wiederholbar ist, versuche dies erneut nach zurücksetzen der Datenbasis.
\item[aufgegeben] eine gescheiterte Transaktion ist hoffnungslos und alle Änderungen werden zurückgesetzt. Danach wird in den Zustand aufgegeben gewechselt.
\end{description}