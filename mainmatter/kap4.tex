\chapter{SQL}
\section{Datentypen}
\begin{lstlisting}
date			// Datum
char(x)			// Zeichenkette wird mit Leerzeichen aufgefuellt
varchar(x)		// Zeichenkette wird NICHT aufgefuellt
numeric(p,s)		// Zahl p = totale laenge s =  # nachkommastellen
integer / int
float
blob / raw		// fuer binaer Dateien
xml			// fuer XML Dateien
\end{lstlisting}

\section{Schemadefinition CREATE TABLE}
\begin{lstlisting}
CREATE TABLE Professoren(
  PersNr integer primary key,
  Name varchar(30) NOT NULL,
  Rang char(2) check (Rang in ('C2', 'C3', 'C4')),
  Raum integer unique,
);
CREATE TABLE pruefen(
  MatrNr integer references Studenten on delete cascade,
  VarlNr integer references Vorlesungen,
  PersNr integer,
  Note numer(2,1) check (Note between 0.7 and 5.0),
  primary key (MatrNr, VorlNr),
  foreign key (PersNr) references Professoren on delete set null
  constraint VorherHoeren
  	check(exists(select *
  			from hoeren h
  			where h.VorlNr = pruefen.VorlNr and
  			      h.MatrNr = pruefen.MatrNr))
);
\end{lstlisting}
\qquad \\
\section{Schemaveränderungen ALTER}
\begin{lstlisting}
ALTER TABLE tabellenname 
	add (Attribut Datentyp);
	add column Attribut Datentyp;
	modify (Attribut Datentyp); // Bedingungen wie NOT NULL bleiben erhalten.
	alter column Attribut Datentyp;
\end{lstlisting}

\section{Einfügen INSERT INTO}
\begin{lstlisting}
INSERT INTO Studenten (MatrNr, Name)
	values (28121, 'Archimedes');

INSERT INTO hoeren
	SELECT MatrNr, VorlNr
	FROM Studenten, Vorlesungen
	WHERE Titel = 'Logik';
\end{lstlisting}
\qquad\\
\section{Datenmanipulation DELETE UPDATE}
\begin{lstlisting}
DELETE FROM Studenten
	WHERE Semester > 13;
	
DELETE FROM voraussetzen
	WHERE Vorgaenger in (SELECT Nachfolger 
			      FROM voraussetzen);

UPDATE Studenten 
	SET Semester = Semester +1;
\end{lstlisting}
\section{Anfragen SELECT}
\begin{lstlisting}
SELECT DISTINCT Rang // zeige elemente der select Menge
	FROM Professoren;

SELECT PersNr, Name
	FROM Professoren
	WHERE Rang = 'C4'
	order by Rang desc, Name asc
\end{lstlisting}
SELECT = $\Pi$ \qquad FROM = $\times$ \qquad WHERE = $\sigma$\\
\newpage
\subsection{Joins}
\begin{lstlisting}
Implizite:

SELECT s.Name
FROM Studenten s, hoeren h, Vorlesungen v, Professoren p
WHERE s.MatrNr = h.MatrNr AND h.VorlNr = v.VorlNr 
AND p.PersNr = v.gelesenVon AND p.Name = 'Curie';

Explizit:

SELECT p.PersNr, p.Name, f.PersNr, f.Note, f.MatrNr, s.MatrNr, s.Name
	from Professoren  p join ( pruefen f join Studenten s 
	on f.MatrNr = s.MatrNr)
	on p.PersNr = f.PersNr;
\end{lstlisting}
\subsection{Gruppierung und Aggregation}
Gruppierung von Werten mit einem Wert durch avg, max, min, sum, count
\begin{lstlisting}
SELECT gelesenVon, Name, sum(sws)
	FROM Vorlesungen
	WHERE gelesenVon = PersNr and Rang = 'C4'
	Group by gelesenVon, Name
		Having avg(SWS) >= 3;
\end{lstlisting}
In der Select-Klausel dürfen nur Aggregatfunktionen oder Attribute, nach denen gruppiert wurde verwendet werden. Somit muss Name auch in die Group by - Klausel (weil nur ein Tupel pro Gruppe vorhanden ist)
\subsection{Geschachtelte Anfragen}
Zu unterscheiden sind dabei Anfragen, die nur ein Tupel zurückliefern und Anfragen, die mehrere Tupel liefern. \\Wird nur ein Tupel geliefert kann in Select-klausel oder where-Klausel Schachtelung verwendet werden, bei denen ein skalarer Wert gefordert wird.\\
\begin{lstlisting}
Select *
From pruefen
Where Note = ( select avg(Note)
		From pruefen);

Select PersNr, Name, (	select sum(SWS) as Lehrbelastung
			From Vorlesungen
			Where gelesenVon = PersNr)
From Professoren
\end{lstlisting}
\begin{description}
\item[Erste Anfrage:] ist eigenständig, verwendet nur ihre eigenen Attribute, sie wird nur einmal ausgewertet
\item[Zweite Anfrage:] ist korreliert, sie verwendet Attribute aus ihrer äußeren Hülle und muss so für jedes Tupel neu durchgeführt werden (für jedes zu überprüfende Tupel, wenn Unteranfrage in where-Bedingung steht und jedes auszugebende Tupel, wenn sie in select-klausel steht)
\end{description}

Noch ein Beispiel:
\begin{lstlisting}

Select s.*			// korreliert
From Studenten
Where exists (select p.*
		From Professoren
		Where p.GebDatum < s.GebDatum);

Select s.*			// unkorreliert
From Studenten
Where s.GebDatum < (select max(p.Gebdatum)
			From Professoren p);
\end{lstlisting}
Noch ein Beispiel (nur durch Join kann Anfrage unkorreliert werden):
\begin{lstlisting}
Select a.*
From Assistenten
Where exists (	select p.*
		From Professoren
		Where p.PersNr = a.Boss and p.GebDatum > a.GebDatum);

Select a.*
From Assistenten a, Professoren p
Where a.GebDatum < p.GebDatum and a.Boss = p.PersNr;
\end{lstlisting}
\subsection{Schachtelung im From-Teil (Bilden einer neuen Relation):}
\begin{lstlisting}
Select tmp.MatrNr, tmp.Name, tmp.VorlAnzahl
	From (	select s.MatrNr, s.Name, count(*) as VorlAnzahl
		From Studenten s, hoeren h
		Where s.MatrNr = h.MatrNr
		Group by s.MatrNr, s.Name) tmp
	Where tmp.VorlAnzahl > 2;



Select h.VorlNr, h.StudProVorl, g.GesStud, h.StudProVorl / 
g.GesStud as Marktanteil
	From 	(select VorlNr, count(*) as StudProVorl
			From hoeren h
			Group by VorlNr; ) h,
		    (select count(*) as GesStud
			From Studenten s ) g;
\end{lstlisting}
\newpage
\subsection{Modularisierung von Anfragen with}
Das Keyword with generiert temporäre Sichten für die Dauer der Abfragebearbeitung. Dadurch können Anfragen übersichtlicher werden. 
\begin{lstlisting}
with h as (select VorlNr, count(*) as StudProVorl
			From hoeren h
			Group by VorlNr )
     g as (select count(*) as GesStud
				From Studenten s )

Select h.VorlNr, h.StudProVorl, g.GesStud, h.StudProVorl / 
g.GesStud as Marktanteil
	From h, g;
\end{lstlisting}

\subsection{Mengenoperationen union, intersect, except}
\begin{lstlisting}
(SELECT Name			
 FROM Assistenten)			
UNION
(SELECT Name			
 FROM Professoren);
 
 SELECT Name
 FROM Professoren
 WHERE PersNr not in (SELECT gelesenVon
 		from Vorlesungen);
\end{lstlisting}

\subsection{Quantifizierte Anfrage}
\begin{itemize}
\item in ist äquivalent zu any
\item any und all sind quantifizierende Bedingungen mit Vergleichsoperatoren
\item any testet, ob es mindestens ein Element im Ergebnis der Unteranfrage gibt, für das der Vergleich mit linkem Argument erfüllt ist
\item all prüft ob alle Elemente die Bedingung erfüllen -> allerdings ist all nicht der Allquantor aus den beschreibenden Sprachen
\end{itemize}
\begin{lstlisting}
Select Name
from Studenten 
where Semester >= all ( 
	select Semester
	from Studenten );	// effizienter waere hier der max-Operator
\end{lstlisting}
\subsubsection{Existenzquantifizierer}
\begin{lstlisting}
select Name
from Professoren
where not exists (select *
		from Vorlesungen
		where gelesenVon = PersNr);
\end{lstlisting}
\subsubsection{Allquantifizierung}
Da es leider keinen Allquantor in SQl gibt, müssen solche Anfragen über einen Existenzquantor ausgedrückt werden. Nachfolgend ein Beispiel \\
\qquad\\
$\{s~|~s\in Studenten \wedge \forall v \in Vorlesungen ( v.SWS = 4\Rightarrow \\ \exists h \in hoeren ( h.VorlNr = v. VorlNr \wedge h.MatrNr = s.MatrNr)) \}$\\
\qquad\\
Umformen mit Hilfe von:\\
\[\begin{array}{lcl}
\forall t\in R(P(t)) &=& \neg(\exists t \in R(\neg P(t))) \\
R \Rightarrow T &=& \neg R \vee T\\
\end{array}\]
$\{s~|~s\in Studenten \wedge \neg (\exists v \in Vorlesungen \neg(\neg( v.SWS = 4 )\vee \\ \exists h \in hoeren ( h.VorlNr = v. VorlNr \wedge h.MatrNr = s.MatrNr))) \}$\\
\qquad \\
Mit der Regel von De-Morgan die Negation nach innen ziehen:\\
\qquad\\
$\{s~|~s\in Studenten \wedge \neg (\exists v \in Vorlesungen ( v.SWS = 4 \wedge \\ \neg(\exists h \in hoeren ( h.VorlNr = v. VorlNr \wedge h.MatrNr = s.MatrNr)))) \}$\\
\section{Null-Werte}
Null wird zu unknown ausgewertet und hat teilweise ein seltsames Verhalten.\\

\begin{tabular}{c|c}
\textbf{not}  &  \\ 
\hline true & false \\ 
unknown & unknown \\ 
 false & true \\ 
\end{tabular} 
\\\qquad\\
\\\qquad\\
\begin{tabular}{c|c|c|c}
\textbf{and}  & true & unknown & false \\ 
\hline true & unknown & false & false \\ 
unknown & unknown & unknown & false \\ 
 false & false & false & false \\ 
\end{tabular} 
\\\qquad\\
\\\qquad\\
\begin{tabular}{c|c|c|c}
 \textbf{or} & true & unknown & false \\ 
\hline true & true & true & true \\ 
 unknown & true & unknown & unknown \\ 
 false & true & unknown & false \\ 
\end{tabular} 
\section{Spezielle Sprachkonstrukte}
\begin{lstlisting}
select *
	from Studenten
	where Semester between 1 and 4;
\end{lstlisting}
\newpage
\begin{lstlisting}
select *
	from Studenten
	where Semester in (1,2,3,4);

select *
	from Studenten
	where Name like 'T%eophrastos';

select Matr, ( case when Note < 1.5 then 'sehr gut'
		when Note < 2.5 then 'gut'
		when Note < 3.5 then 'befriedigend'
		when Note <= 4.0 then 'ausreichend'
		else 'nicht bestanden' end)
	from pruefen;
\end{lstlisting}
\section{Rekursionen}
Rekursionen sind im aktuellen SQL 92 Standard nicht integriert und müssen daher immer explizit komplett ausgeschrieben werden. wie z.B.
\begin{lstlisting}
SELECT v1.Vorgaenger
	FROM voraussetzen v1, voraussetzen v2, Vorlesungen v
	WHERE v1.Nachfolger = v2.Vorgaenger and
		v2.Nachfolger = v.VorlNr and
		v.Titel = 'Der Wiener Kreis';
\end{lstlisting}
\section{Sichten}
Sichten sind sehr nützlich, um Standardabfragen, welche über mehrere Tabellen gehen, welche sich nur sehr selten ändern zu verschnellern. Beispiel:
\begin{lstlisting}
CREATE VIEW pruefenSicht as
	SELECT MatrNr, VorlNr, PersNr
	FROM pruefen;
\end{lstlisting}