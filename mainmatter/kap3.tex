\chapter{Das relationale Modell}
\section{relationales Schema}
Darstellung von Entitytypen:

Student:\{[\underline{MatNr: integer}, Name:string, Semester: integer]\}\\
\qquad\\
Allgemein:

Entityname (Tabellenname) : \{[ x:y | x = Attributbeschreibung; y = Datentyp]\}\\
\qquad \\
Bei dem Initialen Entwurf ist jedwede Relation und Entität in diesem Schema vorhanden. Im nächsten Schritt werden alle Relationen mit dem gleichen Schlüssel zusammengefasst, dabei gilt es zu beachten, das 1:1; 1:N; N:1 Beziehungen mit Fremdschlüssel zusammen gefasst werden können. Auch sollte nach Möglichkeit Nullwerte vermieden werden. 

\section{Relationale Algebra}
\subsection{Selektion $\sigma$}
Bei der \textit{Selektion} werden diejenigen Tupel einer Relation ausgewählt, die das sogenannte \textit{Selektionsprädikat} erfüllen.\\\qquad\\
$\sigma_{\text{Semester>10}}$(Studenten)
\qquad\\
\begin{lstlisting}
SELECT * 
FROM Student
WHERE Semester > 10; 
\end{lstlisting}


\subsection{Projektion $\Pi$}
Extraktion von Attributen von einer Menge\\
\qquad\\
$\Pi_{\text{Rang}}$(Professoren)\\
\begin{lstlisting}
SELECT Rang
FROM Professoren
\end{lstlisting}

\subsection{Kartesisches Produkt $\times$}
$R\times S$ enthält alle möglichen Paare von Tupel aus $R$ und $S$.
\begin{lstlisting}
SELECT * 
FROM Professoren, Assistenten
\end{lstlisting}
\subsection{Umbenennen $\rho$}
$\rho$ wird benötigt, sofern eine Relation mehrfach verwendet wird. \qquad\\
$\Pi_{\text{V1.Vorgänger}}(\sigma_{\text{V2.Nachfolger = 5216}\wedge\text{V1.Nachfolger = V2.Vorgänger}}(\rho_{\text{V1}}(\text{vorraussetzen})\times\rho_{\text{V2}}(\text{vorraussetzen})))$ \\
indirekter Vorgänger von Vorlesung 5216\\
\qquad\\
Attributumbenennung:\\
$\rho_{\text{Vorraussetzung} \leftarrow \text{Vorgänger}}$(voraussetzen)
\begin{lstlisting}
SELECT Vorgaenger AS Vorraussetzung
FROM voraussetzen;
\end{lstlisting}
\subsection{weitere Möglichkeiten bei gleichen Schema}
\subsubsection{Vereinigung $\cup$ UNION}
Vereinigt 2 Mengen, welche projiziert wurden. 
Union hat automatisch Duplikateliminierung! Mit union all vermeidbar. Nur Zeichenketten mit Zeichenketten vereinen.
\begin{lstlisting}
(SELECT Name			
 FROM Assistenten)			
UNION
(SELECT Name			
 FROM Professoren);
\end{lstlisting}
\subsubsection{Mengendifferenz $-$ EXCEPT}
Es wird Tupelmenge aus S von Tupelmenge aus R abgezogen, z. B. Matrikelnummer der Studenten subtrahiert um die Matrikelnummer der Studenten, die Prüfungen geschrieben haben -> wer hat noch keine Prüfung abgelegt?

\subsubsection{Schnittmenge $\cap$ INTERSECT}
Schnittmenge der Tupelmenge zweier Relationen mit gleichem Schema $rightarrow$ Selektion der Tupel, die beide Relationen enthalten.\\
$\Pi_{\text{PersNr}}(\rho_{\text{PersNr}\leftarrow\text{gelesenVon}}(\text{Vorlesungen}))\cap\Pi_{\text{PersNr}}(\sigma_{\text{Rang}=\text{C4}}(\text{Professoren}))$

\subsubsection{Division $\div$ }
$R\div S$ \\
Es bleiben alle Elemente von $R$ übrig die vollständig gejoint werden können mit $S$.

\subsubsection{Gruppierung und Aggregation $\gamma$}
$\gamma_{\text{Semester; \textbf{COUNT(*)}}}(\text{Studenten})$\\
Zählt die Studenten basierend auf deren Semesterwert. Also, wie viele Studenten gibt es in den entsprechenden Semestern. 

\subsection{Joins}
\subsubsection{natürlicher Join $\bowtie$}
$R \bowtie S$\\
Kreuzprodukt über Gleichnamige Attributspalten
\subsubsection{Theta-Join $\bowtie_{\Theta}$}
$R \bowtie_{\Theta} S$\\
$\Theta$ wird ersetzt mit einer Selektion, Projektion oder Umbenennung. Damit die Tabellen entsprechend vor dem Join angepasst und verkleinert werden. 
\subsubsection{linker äußerer Join}
$R ~{\tiny \textifsym{d|><|}}~ S$\\
Alles aus $R$ bleibt erhalten und Fehlende Kombinationen werden mit Null aufgefüllt. 
\subsubsection{rechter äußerer Join}
$R ~{\tiny \textifsym{|><|d}}~ S$\\
Alles aus $S$ bleibt erhalten und Fehlende Kombinationen werden mit Null aufgefüllt. 
\subsubsection{äußerer Join}
$R ~{\tiny \textifsym{d|><|d}}~ S$\\
Alles aus $R$ und $S$ bleibt erhalten und Fehlende Kombinationen werden mit Null aufgefüllt. 
\subsubsection{Semi-Join}
$R \ltimes S$\\
Passende Elemente aus $R$ bleiben erhalten und nur aus $R$ \\
\qquad\\
$R \rtimes S$\\
Passende Elemente aus $S$ bleiben erhalten und nur aus $S$
\subsubsection{Anti-Semi Join}
$R \triangleright S$\\
NICHT passende Elemente aus $R$ bleiben erhalten und nur die aus $R$. \\
\qquad\\
$R \triangleleft S$\\
NICHT passende Elemente aus $S$ bleiben erhalten und nur die aus $S$. 
\section{Tupelkalkül}
generische Form: $\{t|P(t)\}$\\
Um sichere Anfragen gestalten zu können benötigt man eine Domäne. Dies schränkt die Menge auf eine endliche Menge ein.\\
Anfrage auf alle C4 Professoren: $\{p~|~p\in \text{Professoren} \wedge \underbrace{\text{p.Rang = 'C4'}}_{\text{Domäne}}\}$
\begin{lstlisting}
SELECT * 
FROM Professoren
WHERE Rang = 'C4'
\end{lstlisting}
\begin{enumerate}
\item Das Tupel $p$ muss in der Relation $Professoren$ enthalten sein.
\item Das Tupel $p$ muss für das Attribut $Rang$ den Wert 'C4' besitzen. 
\end{enumerate}
Anfragen können auch mit Quantifizierungen gestellt werden.\\
\qquad\\
$\exists t \in R(Q(t))$ bzw. $\forall t \in R(Q(t))$\\
Studenten, welche mindestens eine Vorlesung bei 'Curie' gehört haben.\\
$\{s~|~s \in \text{Studenten}\\
\text{\qquad}\wedge \exists h \in \text{hören}(\text{s.MartNr=h.MatrNr}\\
\text{\qquad } \wedge \exists v \in \text{Vorlesungen}(\text{h.VorlNr=v.VorlNr}\\
\text{\qquad } \wedge \exists p \in \text{Professoren}(\text{v.PersNr=v.gelesenVon}\\
\text{\qquad } \wedge \text{p.Name = 'Curie'})))\}$
\begin{lstlisting}
SELECT s.Name
FROM Studenten s, hoeren h, Vorlesungen v, Professoren p
WHERE s.MatrNr = h.MatrNr AND h.VorlNr = v.VorlNr 
AND p.PersNr = v.gelesenVon AND p.Name = 'Curie';
\end{lstlisting}
\section{Domänenkalkül}
Struktur: $\{[v_1,v_2,\dots,v_n]~|~P(v_1,\dots,v_n)\}$\\
\qquad\\
$v_1,v_2,\dots,v_n$ sind die Attribute welche man erhalten will. $P(v_1,\dots,v_n)$ ist die genaue Anfragenkonstruktion, hier müssen alle Attribute eine eindeutige Variable bekommen. \\
\qquad\\
Implizite Angabe der Joins über die gleiche Variablenbelegung. explizit durch Gleichsetzung der Impliziten Variablen. \\
\qquad\\
$\{[n] ~ | ~ \exists s([m,n,s]\in Studenten \\ 
\text{\qquad}\wedge \exists v([m,v]\in hoeren \\
\text{\qquad}\wedge \exists p ([v,t,sw,p] \in Vorlesungen \\
\text{\qquad}\wedge \exists a ([p,a,r,b] \in Professoren \wedge a = 'Curie'))))\}$

\section{Quantoren Umwandlung}
$\forall t\in R(P(t)) = \neg(\exists t \in R(\neg P(t)))$\\
$\exists t\in R(P(t)) = \neg(\forall t \in R(\neg P(t)))$