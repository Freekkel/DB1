\chapter{Allgemeines}
\section{Klausurtermin}
Die Klausur findet am 14.08.2014 in Raum N1 und/oder N3 statt.

\section{Material für die Klausur}
Bis zum aktuellen Zeitpunkt sind keine Hilfsmittel zugelassen
%
\section{Empfohlene Lektüre:}
Für die Vorlesung wird das folgende Buch dringend empfohlen, da sich die Vorlesung an diesem orientieren wird. \\
Datenbanksysteme: Eine Einführung\\
 Autoren: Alfons Kemper, André Eickler\\
 Auflage: 9.\\
 Oldenbourg Wissenschaftsverlag\\
(25. September 2013)\\
ISBN-10: 3486721399\\
ISBN-13: 978-3486721393\\

Die Auflage Nummer 8 wird für die Übungen verwendet. Auch Auflage 7 kann noch ausreichend sein. Jedoch ist dieses Buch äußerst lange aktuell und es lohnt sich dieses anzuschaffen. In dieser Vorlesung wird Kapitel 1 - 9 behandelt, wobei ein paar Themenbereiche wie z.B. JDBC ausgelassen werden. 

\subsection{Weiterführende Lektüre/Links:}
\begin{enumerate}
\item A. Silberschatz, H. F. Korth und S. Sudarshan Database System Concepts, 4. Auflage, McGraw-Hill Book Co., 2002.
\item R. Elmasri, S.B. Navathe: Fundamentals of Database Systems, Benjamin Cummings, Redwood City, Ca, USA, 2. Auflage, 1994
\item R. Ramakrishnan, J. Gehrke: Database Management Systems, 3. Auflage, 2003.
\item G. Vossen : Datenmodelle, Datenbanksprachen und Datenbank-Management-Systeme. Oldenbourg, 2001.
\item \textbf{D. Maier: The Theory of Relational Databases. Computer Science Press. 1983.}
\item S. M. Lang, P.C. Lockemann: Datenbankeinsatz. Springer Verlage, 1995.
\item C. Batini, S. Ceri, S.B. Navathe: Conceptual Database Design, Benjamin Cummings, Redwood City, Ca, USA, 1992.
\item \textbf{C. J. Date: An Introduction to Database Systems. McGraw-Hill, 8. Aufl., 2003.}
\item J.D. Ullmann, J. Widom: A First Course in Database Systems, McGraw Hill, 2. Auflage, 2001. 8
\item A. Kemper, G. Moerkotte: Object-Oriented Database Management: Applications in Engineering and Computer
Science, Prentice Hall, 1994 
\item E. Rahm: Mehrrechner-Datenbanksyseme. Addison-Wesley, 1994.
\item P. Dadam: Verteilte Datenbanken und Client/Server Systeme. Springer Verlag, 1996
\item G. Weikum, G. Vossen: Transactional Information Systems: Theory, Algorithms, and the Practice of Concurrency Control. Morgan Kaufmann, 2001.
\item T. Härder, E. Rahm: Datenbanksysteme – Konzepte und Techniken der Implementierung, 2001.

\end{enumerate}
\section{Verwendete Software:}
Es wird vorzugsweise auf \href{http://www.postgresql.org/}{PostgreSQL} und \href{http://www.mysql.de/}{MySQL} als OpenSource Varianten zurück gegriffen. Um die Blätter auch ordentlich bearbeiten zu können wird die Installation dieser Systeme empfohlen. In den Prüfungen wurde SQLite verwendet.